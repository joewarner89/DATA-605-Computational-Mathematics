% Options for packages loaded elsewhere
\PassOptionsToPackage{unicode}{hyperref}
\PassOptionsToPackage{hyphens}{url}
%
\documentclass[
]{article}
\usepackage{amsmath,amssymb}
\usepackage{iftex}
\ifPDFTeX
  \usepackage[T1]{fontenc}
  \usepackage[utf8]{inputenc}
  \usepackage{textcomp} % provide euro and other symbols
\else % if luatex or xetex
  \usepackage{unicode-math} % this also loads fontspec
  \defaultfontfeatures{Scale=MatchLowercase}
  \defaultfontfeatures[\rmfamily]{Ligatures=TeX,Scale=1}
\fi
\usepackage{lmodern}
\ifPDFTeX\else
  % xetex/luatex font selection
\fi
% Use upquote if available, for straight quotes in verbatim environments
\IfFileExists{upquote.sty}{\usepackage{upquote}}{}
\IfFileExists{microtype.sty}{% use microtype if available
  \usepackage[]{microtype}
  \UseMicrotypeSet[protrusion]{basicmath} % disable protrusion for tt fonts
}{}
\makeatletter
\@ifundefined{KOMAClassName}{% if non-KOMA class
  \IfFileExists{parskip.sty}{%
    \usepackage{parskip}
  }{% else
    \setlength{\parindent}{0pt}
    \setlength{\parskip}{6pt plus 2pt minus 1pt}}
}{% if KOMA class
  \KOMAoptions{parskip=half}}
\makeatother
\usepackage{xcolor}
\usepackage[margin=1in]{geometry}
\usepackage{color}
\usepackage{fancyvrb}
\newcommand{\VerbBar}{|}
\newcommand{\VERB}{\Verb[commandchars=\\\{\}]}
\DefineVerbatimEnvironment{Highlighting}{Verbatim}{commandchars=\\\{\}}
% Add ',fontsize=\small' for more characters per line
\usepackage{framed}
\definecolor{shadecolor}{RGB}{248,248,248}
\newenvironment{Shaded}{\begin{snugshade}}{\end{snugshade}}
\newcommand{\AlertTok}[1]{\textcolor[rgb]{0.94,0.16,0.16}{#1}}
\newcommand{\AnnotationTok}[1]{\textcolor[rgb]{0.56,0.35,0.01}{\textbf{\textit{#1}}}}
\newcommand{\AttributeTok}[1]{\textcolor[rgb]{0.13,0.29,0.53}{#1}}
\newcommand{\BaseNTok}[1]{\textcolor[rgb]{0.00,0.00,0.81}{#1}}
\newcommand{\BuiltInTok}[1]{#1}
\newcommand{\CharTok}[1]{\textcolor[rgb]{0.31,0.60,0.02}{#1}}
\newcommand{\CommentTok}[1]{\textcolor[rgb]{0.56,0.35,0.01}{\textit{#1}}}
\newcommand{\CommentVarTok}[1]{\textcolor[rgb]{0.56,0.35,0.01}{\textbf{\textit{#1}}}}
\newcommand{\ConstantTok}[1]{\textcolor[rgb]{0.56,0.35,0.01}{#1}}
\newcommand{\ControlFlowTok}[1]{\textcolor[rgb]{0.13,0.29,0.53}{\textbf{#1}}}
\newcommand{\DataTypeTok}[1]{\textcolor[rgb]{0.13,0.29,0.53}{#1}}
\newcommand{\DecValTok}[1]{\textcolor[rgb]{0.00,0.00,0.81}{#1}}
\newcommand{\DocumentationTok}[1]{\textcolor[rgb]{0.56,0.35,0.01}{\textbf{\textit{#1}}}}
\newcommand{\ErrorTok}[1]{\textcolor[rgb]{0.64,0.00,0.00}{\textbf{#1}}}
\newcommand{\ExtensionTok}[1]{#1}
\newcommand{\FloatTok}[1]{\textcolor[rgb]{0.00,0.00,0.81}{#1}}
\newcommand{\FunctionTok}[1]{\textcolor[rgb]{0.13,0.29,0.53}{\textbf{#1}}}
\newcommand{\ImportTok}[1]{#1}
\newcommand{\InformationTok}[1]{\textcolor[rgb]{0.56,0.35,0.01}{\textbf{\textit{#1}}}}
\newcommand{\KeywordTok}[1]{\textcolor[rgb]{0.13,0.29,0.53}{\textbf{#1}}}
\newcommand{\NormalTok}[1]{#1}
\newcommand{\OperatorTok}[1]{\textcolor[rgb]{0.81,0.36,0.00}{\textbf{#1}}}
\newcommand{\OtherTok}[1]{\textcolor[rgb]{0.56,0.35,0.01}{#1}}
\newcommand{\PreprocessorTok}[1]{\textcolor[rgb]{0.56,0.35,0.01}{\textit{#1}}}
\newcommand{\RegionMarkerTok}[1]{#1}
\newcommand{\SpecialCharTok}[1]{\textcolor[rgb]{0.81,0.36,0.00}{\textbf{#1}}}
\newcommand{\SpecialStringTok}[1]{\textcolor[rgb]{0.31,0.60,0.02}{#1}}
\newcommand{\StringTok}[1]{\textcolor[rgb]{0.31,0.60,0.02}{#1}}
\newcommand{\VariableTok}[1]{\textcolor[rgb]{0.00,0.00,0.00}{#1}}
\newcommand{\VerbatimStringTok}[1]{\textcolor[rgb]{0.31,0.60,0.02}{#1}}
\newcommand{\WarningTok}[1]{\textcolor[rgb]{0.56,0.35,0.01}{\textbf{\textit{#1}}}}
\usepackage{graphicx}
\makeatletter
\def\maxwidth{\ifdim\Gin@nat@width>\linewidth\linewidth\else\Gin@nat@width\fi}
\def\maxheight{\ifdim\Gin@nat@height>\textheight\textheight\else\Gin@nat@height\fi}
\makeatother
% Scale images if necessary, so that they will not overflow the page
% margins by default, and it is still possible to overwrite the defaults
% using explicit options in \includegraphics[width, height, ...]{}
\setkeys{Gin}{width=\maxwidth,height=\maxheight,keepaspectratio}
% Set default figure placement to htbp
\makeatletter
\def\fps@figure{htbp}
\makeatother
\setlength{\emergencystretch}{3em} % prevent overfull lines
\providecommand{\tightlist}{%
  \setlength{\itemsep}{0pt}\setlength{\parskip}{0pt}}
\setcounter{secnumdepth}{-\maxdimen} % remove section numbering
\ifLuaTeX
  \usepackage{selnolig}  % disable illegal ligatures
\fi
\IfFileExists{bookmark.sty}{\usepackage{bookmark}}{\usepackage{hyperref}}
\IfFileExists{xurl.sty}{\usepackage{xurl}}{} % add URL line breaks if available
\urlstyle{same}
\hypersetup{
  pdftitle={DATA 605 Assignment 9},
  pdfauthor={Warner Alexis},
  hidelinks,
  pdfcreator={LaTeX via pandoc}}

\title{DATA 605 Assignment 9}
\author{Warner Alexis}
\date{2024-03-25}

\begin{document}
\maketitle

\hypertarget{excercise-11-page-363}{%
\subsection{Excercise 11 page 363}\label{excercise-11-page-363}}

A tourist in Las Vegas was attracted by a certain gambling game in which
the customer stakes 1 dollar on each play; a win then pays the customer
2 dollars plus the return of her stake, although a loss costs her only
her stake. Las Vegas insiders, and alert students of probability theory,
know that the probability of winning at this game is 1/4. When driven
from the tables by hunger, the tourist had played this game 240 times.
Assuming that no near miracles happened, about how much poorer was the
tourist upon leaving the casino? What is the probability that she lost
no money?

\hypertarget{solution}{%
\subsection{Solution}\label{solution}}

Lets \(X_n = Y_n + 1\) \(Y_n\) is an independent random variables having
a mean \(\mu = 0\)

\begin{Shaded}
\begin{Highlighting}[]
\NormalTok{sd }\OtherTok{\textless{}{-}} \FunctionTok{sqrt}\NormalTok{(}\DecValTok{365}\SpecialCharTok{*}\NormalTok{.}\DecValTok{25}\NormalTok{) }\CommentTok{\#will use this for sd for a, b, \& c}
\NormalTok{q }\OtherTok{\textless{}{-}} \DecValTok{100{-}100}
\NormalTok{pr  }\OtherTok{\textless{}{-}} \FunctionTok{pnorm}\NormalTok{(q, }\DecValTok{0}\NormalTok{, sd, }\AttributeTok{lower.tail =} \ConstantTok{FALSE}\NormalTok{)}
\FunctionTok{cat}\NormalTok{(}\StringTok{"the probability that Y365 is ≥ 100: "}\NormalTok{, pr, }\StringTok{"}\SpecialCharTok{\textbackslash{}n}\StringTok{"}\NormalTok{)}
\end{Highlighting}
\end{Shaded}

\begin{verbatim}
## the probability that Y365 is ≥ 100:  0.5
\end{verbatim}

\begin{Shaded}
\begin{Highlighting}[]
\CommentTok{\# calculating x as : (value {-} mean)/sqrt(n)}
\NormalTok{mean }\OtherTok{\textless{}{-}} \DecValTok{0}
\NormalTok{variance }\OtherTok{\textless{}{-}} \DecValTok{1}\SpecialCharTok{/}\DecValTok{4}
\NormalTok{sd }\OtherTok{\textless{}{-}} \FunctionTok{sqrt}\NormalTok{(variance)}
\NormalTok{n }\OtherTok{\textless{}{-}} \DecValTok{364}

\NormalTok{q2 }\OtherTok{\textless{}{-}}\NormalTok{ (}\DecValTok{110} \SpecialCharTok{{-}} \DecValTok{100}\NormalTok{)}\SpecialCharTok{/}\FunctionTok{sqrt}\NormalTok{(n)}
\FunctionTok{cat}\NormalTok{(}\StringTok{"the probability that Y365 is ≥ 110: "}\NormalTok{,}\FunctionTok{pnorm}\NormalTok{(q2, }\AttributeTok{mean =}\NormalTok{ mean, }\AttributeTok{sd =}\NormalTok{ sd, }\AttributeTok{lower.tail =} \ConstantTok{FALSE}\NormalTok{),}\StringTok{"}\SpecialCharTok{\textbackslash{}n}\StringTok{"}\NormalTok{)}
\end{Highlighting}
\end{Shaded}

\begin{verbatim}
## the probability that Y365 is ≥ 110:  0.1472537
\end{verbatim}

\begin{Shaded}
\begin{Highlighting}[]
\DocumentationTok{\#\# calculating x as : (value {-} mean)/sqrt(n)}
\NormalTok{q3 }\OtherTok{\textless{}{-}}\NormalTok{ (}\DecValTok{120} \SpecialCharTok{{-}} \DecValTok{100}\NormalTok{)}\SpecialCharTok{/}\FunctionTok{sqrt}\NormalTok{(n)}
\FunctionTok{cat}\NormalTok{(}\StringTok{"the probability that Y365 is ≥ 120: "}\NormalTok{,}\FunctionTok{pnorm}\NormalTok{(q3, }\AttributeTok{mean =}\NormalTok{ mean, }\AttributeTok{sd =}\NormalTok{ sd, }\AttributeTok{lower.tail =} \ConstantTok{FALSE}\NormalTok{),}\StringTok{"}\SpecialCharTok{\textbackslash{}n}\StringTok{"}\NormalTok{)}
\end{Highlighting}
\end{Shaded}

\begin{verbatim}
## the probability that Y365 is ≥ 120:  0.01801584
\end{verbatim}

\hypertarget{question-2}{%
\subsection{Question 2}\label{question-2}}

Moment generating function \(Mz(t) = Ee^{tx}f(x)\) Binomial distribution
function: \(P(x) = \sum_{j=0} \dbinom{n}{x}p^xq^{(n-x)}\)
\(P(x) = \sum_{j=0} \dbinom{n}{x}(pe^t)q^{(n-x)}\) \(P(x) = (pe^t)q)^n\)
\(g'(t) = n(pe^t + q)^{n-1}pe^t\) \(g'(0) = n(p + q)^{n-1}p\)
\(g'(0) = np\)

then, \(g′′(0)=n(n−1)p^2+np\) \(Var=n(n−1)p^2+np−(np)^2\)
\(=(n^2−n)(p^2)+np−n^2p^2\) \(=np(1−p)\)

\hypertarget{question-3}{%
\subsection{Question 3}\label{question-3}}

Calculate the expected value and variance of the exponential
distribution using the moment generating function.
\(G(t) = \int_{0}^\infty e^{tx} \lambda e^{- \lambda x}dx\)
\(\frac {\lambda(e^{(t- \lambda)} }{t- \lambda} \vert_0 ^\infty\)
\(\frac {\lambda}{\lambda - t}\)
\(g′(t) = \frac {\lambda}{(\lambda - t)^2}\)
\(g′(0) = \frac {\lambda}{(\lambda)^2}\) \(g′(0) = \frac {1}{\lambda}\)

The second derivative evaluated at 0, and the square of the first
derivative evaluated at 0. \(g"(t) = \frac {2\lambda}{(\lambda - t)^3}\)
\(g′(0) = \frac {2\lambda}{(\lambda)^3}\)
\(g′(0) = \frac {2}{(\lambda)^2}\)
\(Var = \frac {2}{(\lambda)^2} - ( \frac {1}{\lambda} )^2\)
\(Var = \frac {2}{\lambda^2} - \frac {1}{\lambda}\)
\(Var = \frac {1}{\lambda^2}\)

he variance of an exponential distribution is \(\frac {1}{\lambda^2}\)

\end{document}
