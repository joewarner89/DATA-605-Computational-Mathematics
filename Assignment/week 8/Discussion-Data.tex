% Options for packages loaded elsewhere
\PassOptionsToPackage{unicode}{hyperref}
\PassOptionsToPackage{hyphens}{url}
%
\documentclass[
]{article}
\usepackage{amsmath,amssymb}
\usepackage{iftex}
\ifPDFTeX
  \usepackage[T1]{fontenc}
  \usepackage[utf8]{inputenc}
  \usepackage{textcomp} % provide euro and other symbols
\else % if luatex or xetex
  \usepackage{unicode-math} % this also loads fontspec
  \defaultfontfeatures{Scale=MatchLowercase}
  \defaultfontfeatures[\rmfamily]{Ligatures=TeX,Scale=1}
\fi
\usepackage{lmodern}
\ifPDFTeX\else
  % xetex/luatex font selection
\fi
% Use upquote if available, for straight quotes in verbatim environments
\IfFileExists{upquote.sty}{\usepackage{upquote}}{}
\IfFileExists{microtype.sty}{% use microtype if available
  \usepackage[]{microtype}
  \UseMicrotypeSet[protrusion]{basicmath} % disable protrusion for tt fonts
}{}
\makeatletter
\@ifundefined{KOMAClassName}{% if non-KOMA class
  \IfFileExists{parskip.sty}{%
    \usepackage{parskip}
  }{% else
    \setlength{\parindent}{0pt}
    \setlength{\parskip}{6pt plus 2pt minus 1pt}}
}{% if KOMA class
  \KOMAoptions{parskip=half}}
\makeatother
\usepackage{xcolor}
\usepackage[margin=1in]{geometry}
\usepackage{color}
\usepackage{fancyvrb}
\newcommand{\VerbBar}{|}
\newcommand{\VERB}{\Verb[commandchars=\\\{\}]}
\DefineVerbatimEnvironment{Highlighting}{Verbatim}{commandchars=\\\{\}}
% Add ',fontsize=\small' for more characters per line
\usepackage{framed}
\definecolor{shadecolor}{RGB}{248,248,248}
\newenvironment{Shaded}{\begin{snugshade}}{\end{snugshade}}
\newcommand{\AlertTok}[1]{\textcolor[rgb]{0.94,0.16,0.16}{#1}}
\newcommand{\AnnotationTok}[1]{\textcolor[rgb]{0.56,0.35,0.01}{\textbf{\textit{#1}}}}
\newcommand{\AttributeTok}[1]{\textcolor[rgb]{0.13,0.29,0.53}{#1}}
\newcommand{\BaseNTok}[1]{\textcolor[rgb]{0.00,0.00,0.81}{#1}}
\newcommand{\BuiltInTok}[1]{#1}
\newcommand{\CharTok}[1]{\textcolor[rgb]{0.31,0.60,0.02}{#1}}
\newcommand{\CommentTok}[1]{\textcolor[rgb]{0.56,0.35,0.01}{\textit{#1}}}
\newcommand{\CommentVarTok}[1]{\textcolor[rgb]{0.56,0.35,0.01}{\textbf{\textit{#1}}}}
\newcommand{\ConstantTok}[1]{\textcolor[rgb]{0.56,0.35,0.01}{#1}}
\newcommand{\ControlFlowTok}[1]{\textcolor[rgb]{0.13,0.29,0.53}{\textbf{#1}}}
\newcommand{\DataTypeTok}[1]{\textcolor[rgb]{0.13,0.29,0.53}{#1}}
\newcommand{\DecValTok}[1]{\textcolor[rgb]{0.00,0.00,0.81}{#1}}
\newcommand{\DocumentationTok}[1]{\textcolor[rgb]{0.56,0.35,0.01}{\textbf{\textit{#1}}}}
\newcommand{\ErrorTok}[1]{\textcolor[rgb]{0.64,0.00,0.00}{\textbf{#1}}}
\newcommand{\ExtensionTok}[1]{#1}
\newcommand{\FloatTok}[1]{\textcolor[rgb]{0.00,0.00,0.81}{#1}}
\newcommand{\FunctionTok}[1]{\textcolor[rgb]{0.13,0.29,0.53}{\textbf{#1}}}
\newcommand{\ImportTok}[1]{#1}
\newcommand{\InformationTok}[1]{\textcolor[rgb]{0.56,0.35,0.01}{\textbf{\textit{#1}}}}
\newcommand{\KeywordTok}[1]{\textcolor[rgb]{0.13,0.29,0.53}{\textbf{#1}}}
\newcommand{\NormalTok}[1]{#1}
\newcommand{\OperatorTok}[1]{\textcolor[rgb]{0.81,0.36,0.00}{\textbf{#1}}}
\newcommand{\OtherTok}[1]{\textcolor[rgb]{0.56,0.35,0.01}{#1}}
\newcommand{\PreprocessorTok}[1]{\textcolor[rgb]{0.56,0.35,0.01}{\textit{#1}}}
\newcommand{\RegionMarkerTok}[1]{#1}
\newcommand{\SpecialCharTok}[1]{\textcolor[rgb]{0.81,0.36,0.00}{\textbf{#1}}}
\newcommand{\SpecialStringTok}[1]{\textcolor[rgb]{0.31,0.60,0.02}{#1}}
\newcommand{\StringTok}[1]{\textcolor[rgb]{0.31,0.60,0.02}{#1}}
\newcommand{\VariableTok}[1]{\textcolor[rgb]{0.00,0.00,0.00}{#1}}
\newcommand{\VerbatimStringTok}[1]{\textcolor[rgb]{0.31,0.60,0.02}{#1}}
\newcommand{\WarningTok}[1]{\textcolor[rgb]{0.56,0.35,0.01}{\textbf{\textit{#1}}}}
\usepackage{graphicx}
\makeatletter
\def\maxwidth{\ifdim\Gin@nat@width>\linewidth\linewidth\else\Gin@nat@width\fi}
\def\maxheight{\ifdim\Gin@nat@height>\textheight\textheight\else\Gin@nat@height\fi}
\makeatother
% Scale images if necessary, so that they will not overflow the page
% margins by default, and it is still possible to overwrite the defaults
% using explicit options in \includegraphics[width, height, ...]{}
\setkeys{Gin}{width=\maxwidth,height=\maxheight,keepaspectratio}
% Set default figure placement to htbp
\makeatletter
\def\fps@figure{htbp}
\makeatother
\setlength{\emergencystretch}{3em} % prevent overfull lines
\providecommand{\tightlist}{%
  \setlength{\itemsep}{0pt}\setlength{\parskip}{0pt}}
\setcounter{secnumdepth}{-\maxdimen} % remove section numbering
\ifLuaTeX
  \usepackage{selnolig}  % disable illegal ligatures
\fi
\IfFileExists{bookmark.sty}{\usepackage{bookmark}}{\usepackage{hyperref}}
\IfFileExists{xurl.sty}{\usepackage{xurl}}{} % add URL line breaks if available
\urlstyle{same}
\hypersetup{
  pdftitle={Data 605 Assignment Week 8},
  pdfauthor={Warner Alexis},
  hidelinks,
  pdfcreator={LaTeX via pandoc}}

\title{Data 605 Assignment Week 8}
\author{Warner Alexis}
\date{2024-03-18}

\begin{document}
\maketitle

\hypertarget{page-303-number-11}{%
\subsection{page 303 number 11}\label{page-303-number-11}}

A company buys 100 lightbulbs, each of which has an exponential lifetime
of 1000 hours. What is the expected time for the rst of these bulbs to
burn out? (See Exercise 10.)

\textbf{Solution}

Lets \((X_i - X_n)\) be independent variables with parameters
\$\lambda\_i \ldots{}\lambda\_n \$
\(Pr(k|X_k = minX_i...X_n) = \frac {\lambda_k}{\lambda_i ...\lambda_n}\)
\$\lambda\emph{i = \frac {1}{1000}, \sum}\{\lambda\_i\} =\frac{1}{10} \$
\(\mu = 1000, n = 100\)

\begin{Shaded}
\begin{Highlighting}[]
\NormalTok{mu }\OtherTok{=} \DecValTok{1000}
\NormalTok{bulb\_n }\OtherTok{=} \DecValTok{100}

\FunctionTok{cat}\NormalTok{(}\StringTok{"The Expected time for the first of these bulbs to burn out is "}\NormalTok{, mu}\SpecialCharTok{/}\NormalTok{bulb\_n ,}\StringTok{"hours "}\NormalTok{)}
\end{Highlighting}
\end{Shaded}

\begin{verbatim}
## The Expected time for the first of these bulbs to burn out is  10 hours
\end{verbatim}

\hypertarget{page-303-14}{%
\subsection{page 303 \# 14}\label{page-303-14}}

Particles are subject to collisions that cause them to split into two
parts with each part a fraction of the parent. Suppose that this
fraction is uniformly distributed between 0 and 1. Following a single
particle through several splittings we obtain a fraction of the original
particle \(Zn = X_1*X_2...X_n\) where each Xj is uniformly distributed
between 0 and 1. Show that the density for the random variable \(Z_n\)
is

\(fn^(z^) = \frac {1}{(n-1)!}(-log z)^n-1\)

\textbf{Solution}

Lets Z = X1 - X2 when z \textgreater{} 0 \(z = x_1 - x_2\)
\(x_2 = x_1 - z\)
\(fZ(x) = \int_{\infty}^\infty f_x1(x_1) f_x2(x1 - z)dx_1\) Solve it by
exponential distribution \(fX(x) = \lambda e ^{-\lambda x}\)
\(\int_{0}^\infty \lambda ^{2\lambda x_1} \lambda e ^{-\lambda (x_1 - z)} dx_1\)
\(\int_{0}^\infty \lambda ^2e^{\lambda( -2x_1 + \lambda z)} dx_1\)
\(\int_{0}^\infty \lambda ^2e^{\lambda (z-2x_1)}dx_1\)
\((1/2)\lambda 2^{-\lambda(z)}\)

\hypertarget{page-320-321}{%
\subsection{page 320-321}\label{page-320-321}}

A fair coin is tossed 100 times. The expected number of heads is 50, and
the standard deviation for the number of heads is (100  1=2  1=2)1=2 =
5. What does Chebyshev's Inequality tell you about the probability that
the number of heads that turn up deviates from the expected number 50 by
three or more standard deviations (i.e., by at least 15)?

\textbf{Solution}

Suppose that we have the following input:
\(\mu = 10 , variance = 100/3\)
\(P(|X-\mu|) >= k\sigma \le \frac {\sigma^2}{k^2\sigma^2} = \frac {1}{k^2}\)
\$P(\textbar X-10\textbar) \textgreater= 2 \le \frac {1}{2^2} \$

\begin{Shaded}
\begin{Highlighting}[]
\CommentTok{\#MEan }
\NormalTok{mu }\OtherTok{=} \DecValTok{10} 
\CommentTok{\# variance}
\NormalTok{var }\OtherTok{=} \DecValTok{100}\SpecialCharTok{/}\DecValTok{3}
\CommentTok{\# Standard deviation }
\NormalTok{sd }\OtherTok{=} \FunctionTok{sqrt}\NormalTok{(var)}
\NormalTok{ksd }\OtherTok{=} \DecValTok{2} 
\NormalTok{k }\OtherTok{\textless{}{-}}\NormalTok{ ksd }\SpecialCharTok{/}\NormalTok{ sd}

\NormalTok{upper\_bnd }\OtherTok{=} \DecValTok{1}\SpecialCharTok{/}\NormalTok{(k}\SpecialCharTok{\^{}}\DecValTok{2}\NormalTok{)}


\FunctionTok{cat}\NormalTok{(}\StringTok{"The the upper bound is "}\NormalTok{, upper\_bnd,}\StringTok{"}\SpecialCharTok{\textbackslash{}n}\StringTok{"}\NormalTok{)}
\end{Highlighting}
\end{Shaded}

\begin{verbatim}
## The the upper bound is  8.333333
\end{verbatim}

\begin{Shaded}
\begin{Highlighting}[]
\FunctionTok{cat}\NormalTok{(}\StringTok{"The propability of the upper bound is "}\NormalTok{, }\FunctionTok{pmin}\NormalTok{(upper\_bnd))}
\end{Highlighting}
\end{Shaded}

\begin{verbatim}
## The propability of the upper bound is  8.333333
\end{verbatim}

When \(P(|X-10|) >+ 5\) then we will have : \$P(\textbar X-10\textbar)
\textgreater= 5 \le \frac {1}{5^2} \$

\begin{Shaded}
\begin{Highlighting}[]
\NormalTok{sd\_k  }\OtherTok{=} \DecValTok{5} 

\NormalTok{k }\OtherTok{\textless{}{-}} \DecValTok{5} \SpecialCharTok{/}\NormalTok{ sd}
\NormalTok{upper\_bnd }\OtherTok{=} \DecValTok{1}\SpecialCharTok{/}\NormalTok{(k}\SpecialCharTok{\^{}}\DecValTok{2}\NormalTok{)}
\FunctionTok{cat}\NormalTok{(}\StringTok{"The highest propability is  "}\NormalTok{, upper\_bnd,}\StringTok{"}\SpecialCharTok{\textbackslash{}n}\StringTok{"}\NormalTok{)}
\end{Highlighting}
\end{Shaded}

\begin{verbatim}
## The highest propability is   1.333333
\end{verbatim}

\begin{Shaded}
\begin{Highlighting}[]
\FunctionTok{cat}\NormalTok{(}\StringTok{"The propability of the upper bound is "}\NormalTok{, }\FunctionTok{pmin}\NormalTok{(upper\_bnd))}
\end{Highlighting}
\end{Shaded}

\begin{verbatim}
## The propability of the upper bound is  1.333333
\end{verbatim}

When \(P(|X-10|) >= 9\) then we will have : \$P(\textbar X-10\textbar)
\textgreater= 9 \le \frac {1}{9^2} \$

\begin{Shaded}
\begin{Highlighting}[]
\NormalTok{sd\_k  }\OtherTok{=} \DecValTok{9}

\NormalTok{k }\OtherTok{\textless{}{-}}\NormalTok{ sd\_k }\SpecialCharTok{/}\NormalTok{ sd}
\NormalTok{upper\_bnd }\OtherTok{=} \DecValTok{1}\SpecialCharTok{/}\NormalTok{(k}\SpecialCharTok{\^{}}\DecValTok{2}\NormalTok{)}
\FunctionTok{cat}\NormalTok{(}\StringTok{"The highest propability is  "}\NormalTok{, upper\_bnd,}\StringTok{"}\SpecialCharTok{\textbackslash{}n}\StringTok{"}\NormalTok{)}
\end{Highlighting}
\end{Shaded}

\begin{verbatim}
## The highest propability is   0.4115226
\end{verbatim}

\begin{Shaded}
\begin{Highlighting}[]
\FunctionTok{cat}\NormalTok{(}\StringTok{"The propability of the upper bound is "}\NormalTok{, }\FunctionTok{pmin}\NormalTok{(upper\_bnd))}
\end{Highlighting}
\end{Shaded}

\begin{verbatim}
## The propability of the upper bound is  0.4115226
\end{verbatim}

When \(P(|X-10|) >= 20\) then we will have : \$P(\textbar X-10\textbar)
\textgreater= 20 \le \frac {1}{20^2} \$

\begin{Shaded}
\begin{Highlighting}[]
\NormalTok{sd\_k  }\OtherTok{=} \DecValTok{20} 

\NormalTok{k }\OtherTok{\textless{}{-}}\NormalTok{ sd\_k }\SpecialCharTok{/}\NormalTok{ sd}
\NormalTok{upper\_bnd }\OtherTok{=} \DecValTok{1}\SpecialCharTok{/}\NormalTok{(k}\SpecialCharTok{\^{}}\DecValTok{2}\NormalTok{)}
\FunctionTok{cat}\NormalTok{(}\StringTok{"The highest propability is  "}\NormalTok{, upper\_bnd,}\StringTok{"}\SpecialCharTok{\textbackslash{}n}\StringTok{"}\NormalTok{)}
\end{Highlighting}
\end{Shaded}

\begin{verbatim}
## The highest propability is   0.08333333
\end{verbatim}

\begin{Shaded}
\begin{Highlighting}[]
\FunctionTok{cat}\NormalTok{(}\StringTok{"The propability of the upper bound is "}\NormalTok{, }\FunctionTok{pmin}\NormalTok{(upper\_bnd))}
\end{Highlighting}
\end{Shaded}

\begin{verbatim}
## The propability of the upper bound is  0.08333333
\end{verbatim}

\end{document}
