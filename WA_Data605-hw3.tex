% Options for packages loaded elsewhere
\PassOptionsToPackage{unicode}{hyperref}
\PassOptionsToPackage{hyphens}{url}
%
\documentclass[
]{article}
\usepackage{amsmath,amssymb}
\usepackage{iftex}
\ifPDFTeX
  \usepackage[T1]{fontenc}
  \usepackage[utf8]{inputenc}
  \usepackage{textcomp} % provide euro and other symbols
\else % if luatex or xetex
  \usepackage{unicode-math} % this also loads fontspec
  \defaultfontfeatures{Scale=MatchLowercase}
  \defaultfontfeatures[\rmfamily]{Ligatures=TeX,Scale=1}
\fi
\usepackage{lmodern}
\ifPDFTeX\else
  % xetex/luatex font selection
\fi
% Use upquote if available, for straight quotes in verbatim environments
\IfFileExists{upquote.sty}{\usepackage{upquote}}{}
\IfFileExists{microtype.sty}{% use microtype if available
  \usepackage[]{microtype}
  \UseMicrotypeSet[protrusion]{basicmath} % disable protrusion for tt fonts
}{}
\makeatletter
\@ifundefined{KOMAClassName}{% if non-KOMA class
  \IfFileExists{parskip.sty}{%
    \usepackage{parskip}
  }{% else
    \setlength{\parindent}{0pt}
    \setlength{\parskip}{6pt plus 2pt minus 1pt}}
}{% if KOMA class
  \KOMAoptions{parskip=half}}
\makeatother
\usepackage{xcolor}
\usepackage[margin=1in]{geometry}
\usepackage{color}
\usepackage{fancyvrb}
\newcommand{\VerbBar}{|}
\newcommand{\VERB}{\Verb[commandchars=\\\{\}]}
\DefineVerbatimEnvironment{Highlighting}{Verbatim}{commandchars=\\\{\}}
% Add ',fontsize=\small' for more characters per line
\usepackage{framed}
\definecolor{shadecolor}{RGB}{248,248,248}
\newenvironment{Shaded}{\begin{snugshade}}{\end{snugshade}}
\newcommand{\AlertTok}[1]{\textcolor[rgb]{0.94,0.16,0.16}{#1}}
\newcommand{\AnnotationTok}[1]{\textcolor[rgb]{0.56,0.35,0.01}{\textbf{\textit{#1}}}}
\newcommand{\AttributeTok}[1]{\textcolor[rgb]{0.13,0.29,0.53}{#1}}
\newcommand{\BaseNTok}[1]{\textcolor[rgb]{0.00,0.00,0.81}{#1}}
\newcommand{\BuiltInTok}[1]{#1}
\newcommand{\CharTok}[1]{\textcolor[rgb]{0.31,0.60,0.02}{#1}}
\newcommand{\CommentTok}[1]{\textcolor[rgb]{0.56,0.35,0.01}{\textit{#1}}}
\newcommand{\CommentVarTok}[1]{\textcolor[rgb]{0.56,0.35,0.01}{\textbf{\textit{#1}}}}
\newcommand{\ConstantTok}[1]{\textcolor[rgb]{0.56,0.35,0.01}{#1}}
\newcommand{\ControlFlowTok}[1]{\textcolor[rgb]{0.13,0.29,0.53}{\textbf{#1}}}
\newcommand{\DataTypeTok}[1]{\textcolor[rgb]{0.13,0.29,0.53}{#1}}
\newcommand{\DecValTok}[1]{\textcolor[rgb]{0.00,0.00,0.81}{#1}}
\newcommand{\DocumentationTok}[1]{\textcolor[rgb]{0.56,0.35,0.01}{\textbf{\textit{#1}}}}
\newcommand{\ErrorTok}[1]{\textcolor[rgb]{0.64,0.00,0.00}{\textbf{#1}}}
\newcommand{\ExtensionTok}[1]{#1}
\newcommand{\FloatTok}[1]{\textcolor[rgb]{0.00,0.00,0.81}{#1}}
\newcommand{\FunctionTok}[1]{\textcolor[rgb]{0.13,0.29,0.53}{\textbf{#1}}}
\newcommand{\ImportTok}[1]{#1}
\newcommand{\InformationTok}[1]{\textcolor[rgb]{0.56,0.35,0.01}{\textbf{\textit{#1}}}}
\newcommand{\KeywordTok}[1]{\textcolor[rgb]{0.13,0.29,0.53}{\textbf{#1}}}
\newcommand{\NormalTok}[1]{#1}
\newcommand{\OperatorTok}[1]{\textcolor[rgb]{0.81,0.36,0.00}{\textbf{#1}}}
\newcommand{\OtherTok}[1]{\textcolor[rgb]{0.56,0.35,0.01}{#1}}
\newcommand{\PreprocessorTok}[1]{\textcolor[rgb]{0.56,0.35,0.01}{\textit{#1}}}
\newcommand{\RegionMarkerTok}[1]{#1}
\newcommand{\SpecialCharTok}[1]{\textcolor[rgb]{0.81,0.36,0.00}{\textbf{#1}}}
\newcommand{\SpecialStringTok}[1]{\textcolor[rgb]{0.31,0.60,0.02}{#1}}
\newcommand{\StringTok}[1]{\textcolor[rgb]{0.31,0.60,0.02}{#1}}
\newcommand{\VariableTok}[1]{\textcolor[rgb]{0.00,0.00,0.00}{#1}}
\newcommand{\VerbatimStringTok}[1]{\textcolor[rgb]{0.31,0.60,0.02}{#1}}
\newcommand{\WarningTok}[1]{\textcolor[rgb]{0.56,0.35,0.01}{\textbf{\textit{#1}}}}
\usepackage{graphicx}
\makeatletter
\def\maxwidth{\ifdim\Gin@nat@width>\linewidth\linewidth\else\Gin@nat@width\fi}
\def\maxheight{\ifdim\Gin@nat@height>\textheight\textheight\else\Gin@nat@height\fi}
\makeatother
% Scale images if necessary, so that they will not overflow the page
% margins by default, and it is still possible to overwrite the defaults
% using explicit options in \includegraphics[width, height, ...]{}
\setkeys{Gin}{width=\maxwidth,height=\maxheight,keepaspectratio}
% Set default figure placement to htbp
\makeatletter
\def\fps@figure{htbp}
\makeatother
\setlength{\emergencystretch}{3em} % prevent overfull lines
\providecommand{\tightlist}{%
  \setlength{\itemsep}{0pt}\setlength{\parskip}{0pt}}
\setcounter{secnumdepth}{-\maxdimen} % remove section numbering
\ifLuaTeX
  \usepackage{selnolig}  % disable illegal ligatures
\fi
\IfFileExists{bookmark.sty}{\usepackage{bookmark}}{\usepackage{hyperref}}
\IfFileExists{xurl.sty}{\usepackage{xurl}}{} % add URL line breaks if available
\urlstyle{same}
\hypersetup{
  pdftitle={Ass 3},
  pdfauthor={Warner Alexis},
  hidelinks,
  pdfcreator={LaTeX via pandoc}}

\title{Ass 3}
\author{Warner Alexis}
\date{2024-02-12}

\begin{document}
\maketitle

\hypertarget{assignment-3}{%
\subsection{Assignment 3}\label{assignment-3}}

we have a matric A
\(A = \begin{bmatrix}1 & 2 & 3 & 4\\ -1 &0 & 1 & 3 \\ 0 & 1 & -2 & 1 \\ 5 & 4 & -2 & -3\end{bmatrix}\).
We are goijng to reduce the matrix to echelon form R2 \textless- R2 + R1
R4 \textless- R4 - 5R1

that will give us this matrix
\(A = \begin{bmatrix}1 & 2 & 3 & 4\\ 0 &2 & 4 & 7 \\ 0 & 1 & -2 & 1 \\ 0 & -6 & -17 & -23\end{bmatrix}\)

we continue to break down the matrix so we can get the non zero rows

R2 \textless- 1/2R2

R4 \textless- R4 - 6R3

That will give us
\(A = \begin{bmatrix}1 & 2 & 3 & 4\\ 0 &1 & 2 & 7/2 \\ 0 & 0 & -4 & 5/2 \\ 0 & 0 & -5 & -17\end{bmatrix}\)

R3 \textless- R3 + R4

R4 \textless- R4 + R4

Final matrix is
\(A = \begin{bmatrix}1 & 2 & 3 & 4\\ 0 &1 & 2 & 7/2 \\ 0 & 0 & 1 & 24/2 \\ 0 & 0 & 1 & -39/2\end{bmatrix}\)

Therefore, n,r(A) = n . we imply that rank is 4

\begin{Shaded}
\begin{Highlighting}[]
\CommentTok{\# initialize matrix}
\NormalTok{A }\OtherTok{\textless{}{-}} \FunctionTok{matrix}\NormalTok{(}\FunctionTok{c}\NormalTok{(}\DecValTok{1}\NormalTok{,}\DecValTok{2}\NormalTok{,}\DecValTok{3}\NormalTok{,}\DecValTok{4}\NormalTok{,}\SpecialCharTok{{-}}\DecValTok{1}\NormalTok{,}\DecValTok{0}\NormalTok{,}\DecValTok{1}\NormalTok{,}\DecValTok{3}\NormalTok{,}\DecValTok{0}\NormalTok{,}\DecValTok{1}\NormalTok{,}\SpecialCharTok{{-}}\DecValTok{2}\NormalTok{,}\DecValTok{1}\NormalTok{,}\DecValTok{5}\NormalTok{,}\DecValTok{4}\NormalTok{,}\SpecialCharTok{{-}}\DecValTok{2}\NormalTok{,}\SpecialCharTok{{-}}\DecValTok{3}\NormalTok{),}\AttributeTok{nrow=}\DecValTok{4}\NormalTok{,}\AttributeTok{byrow=} \ConstantTok{TRUE}\NormalTok{)}
\NormalTok{A}
\end{Highlighting}
\end{Shaded}

\begin{verbatim}
##      [,1] [,2] [,3] [,4]
## [1,]    1    2    3    4
## [2,]   -1    0    1    3
## [3,]    0    1   -2    1
## [4,]    5    4   -2   -3
\end{verbatim}

\begin{Shaded}
\begin{Highlighting}[]
\CommentTok{\# use matrix library }
\FunctionTok{library}\NormalTok{(Matrix)}
\FunctionTok{rankMatrix}\NormalTok{(A)[}\DecValTok{1}\NormalTok{][}\DecValTok{1}\NormalTok{]}
\end{Highlighting}
\end{Shaded}

\begin{verbatim}
## [1] 4
\end{verbatim}

\#\#2 Given an m x n matrix where m \textgreater{} n, what can be the
maximum rank? The minimum rank, assuming that the matrix is non-zero?

Thee maximum rank a matrix m x n can have the maximum rank of n because
it is not possible to have more than n linearly independent columns. The
minimum rank assuming the matrix is non-zero would be 1

\#3 the rank for matrix

\(B = \begin{bmatrix}1 & 2 & 1\\ 3 &6 & 3 \\ 2 & 4 & 2 \end{bmatrix}\)

We are going to reduce the matrix to echelon form

R2 \textless- R2 -3R1

R3 \textless- R3 - 2R1

We will have this matrix:

\(B = \begin{bmatrix}1 & 2 & 1\\ 0 &0 & 0 \\ 0 & 0 & 0 \end{bmatrix}\)

Therefore, n,r(A) = n . we imply that rank is 1

\begin{Shaded}
\begin{Highlighting}[]
\NormalTok{B }\OtherTok{\textless{}{-}} \FunctionTok{matrix}\NormalTok{(}\FunctionTok{c}\NormalTok{(}\DecValTok{1}\NormalTok{,}\DecValTok{2}\NormalTok{,}\DecValTok{1}\NormalTok{,}\DecValTok{3}\NormalTok{,}\DecValTok{6}\NormalTok{,}\DecValTok{3}\NormalTok{,}\DecValTok{2}\NormalTok{,}\DecValTok{4}\NormalTok{,}\DecValTok{2}\NormalTok{), }\AttributeTok{nrow =} \DecValTok{3}\NormalTok{, }\AttributeTok{ncol =} \DecValTok{3}\NormalTok{, }\AttributeTok{byrow =} \ConstantTok{TRUE}\NormalTok{)}
\NormalTok{B}
\end{Highlighting}
\end{Shaded}

\begin{verbatim}
##      [,1] [,2] [,3]
## [1,]    1    2    1
## [2,]    3    6    3
## [3,]    2    4    2
\end{verbatim}

\begin{Shaded}
\begin{Highlighting}[]
\FunctionTok{rankMatrix}\NormalTok{(B)[}\DecValTok{1}\NormalTok{][}\DecValTok{1}\NormalTok{]}
\end{Highlighting}
\end{Shaded}

\begin{verbatim}
## [1] 1
\end{verbatim}

\hypertarget{q3}{%
\subsection{Q3}\label{q3}}

Lets a matrix
\(B = \begin{bmatrix}1 & 2 & 3\\ 0 &4 & 5 \\ 0 & 0 & 6 \end{bmatrix}\)

find the eigenvalues and eigenvectors:

\hypertarget{step-1}{%
\subsection{Step 1}\label{step-1}}

\(|A - \lambda I | = 0\)

so we have : \$B =

\begin{bmatrix}1 & 2 & 3\\ 0 &4 & 5  \\ 0 & 0 & 6 \end{bmatrix}

\begin{itemize}
\tightlist
\item
  \lambda 

  \begin{bmatrix}1 & 0 & 0\\ 0 &1 & 0  \\ 0 & 0 & 1 \end{bmatrix}

  \$
\end{itemize}

\(B = \begin{bmatrix}1-\lambda & 2 & 3\\ 0 &4-\lambda & 5 \\ 0 & 0 & 6-\lambda \end{bmatrix}\)

\hypertarget{step-2}{%
\subsection{Step 2}\label{step-2}}

\(1 - \lambda \begin{bmatrix}1-\lambda & 5 \\ 0 &6-\lambda \end{bmatrix} - 2\begin{bmatrix}0 & 5 \\ 0 &6-\lambda \end{bmatrix} +3\begin{bmatrix}0 & 4-\lambda \\ 0 &0 \end{bmatrix}\)

\((1 - \lambda) (4-\lambda) (6-\lambda)\)

Eigenvalue will be 6 4 1

\((1 - \lambda) (4-\lambda) (6-\lambda)\)
\((\lambda^3 - 5\lambda^2 +4\lambda)-(6\lambda^2 -30\lambda + 24)\)
\(\lambda^3 - 11\lambda^2 - 34\lambda - 24\)

\begin{Shaded}
\begin{Highlighting}[]
\NormalTok{A }\OtherTok{\textless{}{-}} \FunctionTok{matrix}\NormalTok{(}\FunctionTok{c}\NormalTok{(}\DecValTok{1}\NormalTok{,}\DecValTok{2}\NormalTok{,}\DecValTok{3}\NormalTok{,}\DecValTok{0}\NormalTok{,}\DecValTok{4}\NormalTok{,}\DecValTok{5}\NormalTok{,}\DecValTok{0}\NormalTok{,}\DecValTok{0}\NormalTok{,}\DecValTok{6}\NormalTok{), }\AttributeTok{nrow =} \DecValTok{3}\NormalTok{, }\AttributeTok{ncol =} \DecValTok{3}\NormalTok{, }\AttributeTok{byrow =} \ConstantTok{TRUE}\NormalTok{)}
\FunctionTok{eigen}\NormalTok{(A)}
\end{Highlighting}
\end{Shaded}

\begin{verbatim}
## eigen() decomposition
## $values
## [1] 6 4 1
## 
## $vectors
##           [,1]      [,2] [,3]
## [1,] 0.5108407 0.5547002    1
## [2,] 0.7981886 0.8320503    0
## [3,] 0.3192754 0.0000000    0
\end{verbatim}

Calculate the eigenvector. we find the eigenvectors corresponding to
each eigenvalue by solving the equation \$(A-\lambda)v =0 \$

for \((\lambda - 6)\) :

\(B = \begin{bmatrix}1-6 & 2 & 3\\ 0 &4-6 & 5 \\ 0 & 0 & 6-6 \end{bmatrix}\)

\(R1 <- -1/5R1\) \(R2 <- -1/5R2\)

\(B = \begin{bmatrix}1 & -2/5 & -3/5\\ 0 &1 & 5/2 \\ 0 & 0 & 0 \end{bmatrix}\)

\(v2 - 5/2v3 = 0\) \(v1 - 2/5v2-3/5v3 = 0\) \(v1 = t\)

for \((\lambda - 6) = \begin{bmatrix}t & 5/2 & t \end{bmatrix}\)

for \((\lambda - 4)\) :

\(A = \begin{bmatrix}1-4 & 2 & 3\\ 0 &4-4 & 5 \\ 0 & 0 & 6-4 \end{bmatrix}\)

\(R1 <- -1/3R1\)

\(B = \begin{bmatrix}1 & -2/53 & -1\\ 0 &0 & 5 \\ 0 & 0 & 2 \end{bmatrix}\)

\(5v3 = 0\) \(v1 - 2/3v2-v3 = 0\) \(v1 = -2/3t\)

for \((\lambda - 4) = \begin{bmatrix}-2/3 & 1 & 0 \end{bmatrix}\)

\begin{Shaded}
\begin{Highlighting}[]
\CommentTok{\# calculate the eigenvector}
\NormalTok{A }\OtherTok{\textless{}{-}} \FunctionTok{matrix}\NormalTok{(}\FunctionTok{c}\NormalTok{(}\DecValTok{1}\NormalTok{,}\DecValTok{2}\NormalTok{,}\DecValTok{3}\NormalTok{,}\DecValTok{0}\NormalTok{,}\DecValTok{4}\NormalTok{,}\DecValTok{5}\NormalTok{,}\DecValTok{0}\NormalTok{,}\DecValTok{0}\NormalTok{,}\DecValTok{6}\NormalTok{), }\AttributeTok{nrow =} \DecValTok{3}\NormalTok{, }\AttributeTok{ncol =} \DecValTok{3}\NormalTok{, }\AttributeTok{byrow =} \ConstantTok{TRUE}\NormalTok{)}

\FunctionTok{eigen}\NormalTok{(A)[}\DecValTok{2}\NormalTok{]}
\end{Highlighting}
\end{Shaded}

\begin{verbatim}
## $vectors
##           [,1]      [,2] [,3]
## [1,] 0.5108407 0.5547002    1
## [2,] 0.7981886 0.8320503    0
## [3,] 0.3192754 0.0000000    0
\end{verbatim}

\end{document}
